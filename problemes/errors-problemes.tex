%%%%%%%%%%%%%%%%%%%%%%%%%%%%%%%%%%%%%%%%%%%%%%%%%%%%%%%%%%%%%%%%%%%%%%%%%%%%%%
% Tema 1. Problemes. Aritmètica Finita i Control d'Errors
%%%%%%%%%%%%%%%%%%%%%%%%%%%%%%%%%%%%%%%%%%%%%%%%%%%%%%%%%%%%%%%%%%%%%%%%%%%%%%
\documentclass[a4paper,twoside,12pt]{exam}
\printanswers
\usepackage{problemes}

\def\assignatura{\`{A}lgebra Lineal Num\`{e}rica}
\def\codi{200151}
\def\GrauMaster{MAT}
\def\EscolaFacultat{FME}

\hypersetup{
  pdfauthor   = {Equip Docent ALN},
  pdftitle    = {Problemes Tema 1: Arim\`{e}tica Finita i Control d'Errors},
  pdfsubject  = {Problemes}
}

\renewcommand{\solutiontitle}%
{\noindent\textbf{Solució:}\enspace}

\pagestyle{head}
\firstpageheader{\EscolaFacultat}%
		{\assignatura~(\codi)}%
		{\thepage}%
\firstpageheadrule

\runningheader{}{\thesection.~\currentname}{\thepage}
\runningheadrule

\bracketedpoints
\unframedsolutions

\lstdefinestyle{myc++}{
	language=[11]C++,
  	frame=single,
	numbers=left,
	numberstyle=\small,
%	framexleftmargin=15pt,
%	numbersep=8pt,
	captionpos=b,
	showstringspaces=flase,
	escapeinside={(*}{*)},
	mathescape
}

%\captionsetup[lstlisting]{font=footnotesize}
\lstset{inputpath=../src/}


\renewcommand{\lstlistingname}{Llistat}% Listing -> Llistat
\renewcommand{\lstlistlistingname}%
{Índex dels \lstlistingname s}% List of Listings -> Índex dels llistats

\newcommand{\includecode}[2][c]
 {\lstinputlisting[caption=#2, escapechar=,
   style=custom#1]{#2}<!---->}

% Example
% \includecode{sched.c}
% \includecode[asm]{sched.s}

\newcommand{\osnp}[1]{\oldstyle{\np{#1}}}
\npdecimalsign{.}
\npthousandsep{\,}
\npproductsign{\times}%

\renewcommand{\choicelabel}{(\roman{choice})}

\begin{document}
\section{Aritmètica finita i control d'errors}
%\setcounter{problema}{0}

\begin{questions}
%--------------------------------- Problema 1 --------------------------------
\question\label{prob:1}
Calculeu $f(x) = 1 - \cos(x)$ amb aritmètica de coma flotant de $\osnp{6}$
dígits, per a $|x| < \osnp{e-3}$. És fiable el resultat? Feu el mateix amb la
representació de $f(x)$ 
\begin{displaymath}
	f(x) = \frac{\sin^{2}(x)}{1 + \cos(x)}.
\end{displaymath}
Trobeu una altra representació alternativa per a $f(x)$ que sigui vàlida per a
$|x|$ petit.
\begin{solution}
	Sigui $\osnp{0}\le x \le \osnp{e-3}$. Aleshores $\cos(\osnp{e-3})\le\cos(x)
	\le \cos(\osnp{0}) = \osnp{1}$ i, per tant
\begin{displaymath}
	\fl{A}{\cos(10^{-3})}
	\le \fl{A}{\cos(x)} \le \fl{A}{\osnp{1}} = \osnp{1}.
\end{displaymath}
%\osnprounddigits{6}
Tenim doncs $\fl{A}{\cos(\numprint{e-3})} = \fl{A}{\osnp{0.9999995000}\ldots} = 1$
(6 dígits amb arrodoniment). Aleshores: $\osnp{1}\le\fl{A}{\cos(x)}\le\fl{A}{\osnp{1}} =
\osnp{1}$, amb la qual cosa $\fl{A}{\cos(x)} = \osnp{1}$, per a $|x|\le\osnp{e-3}$. En
conclusió, si fem servir aritmètica de coma flotant de $\osnp{6}$ dígits amb
arrodoniment, llavors:
\begin{displaymath}
	\fl{A}{\osnp{1}-\cos(x)} = \osnp{1} -\fl{A}{\cos(x)} = 
	\osnp{1} - \osnp{1} = \osnp{0}.
\end{displaymath}
Aquesta situació ---on es resten resultats molt semblants---es coneix com
\emph{cance\lgem ació}. Les cance\lgem acions magnifiquen la propagació dels
erros i hem d'evitar-les als nostres càlculs. En aquest cas podem solucionar el
problema de cance\lgem acions re-formulant $f(x)$:
\begin{displaymath}
	f(x) = \left(\osnp{1}-\cos(x)\right)\frac{\osnp{1}+\cos(x)}%
	{\osnp{1} + \cos(x)}
	= \frac{\osnp{1}-\cos^{\osnp{2}}(x)}{\osnp{1}+\cos(x)} 
	= \frac{\sin^{\osnp{2}}(x)}{\osnp{1}+\cos(x)}.
\end{displaymath}
Així per exemple, per $x_{\osnp{0}} = \osnp{e-3}$, tenim:
\begin{align*}
	\fl{A}{\sin^{2}(\osnp{e-3}} &= \fl{A}{\sin(\osnp{e-3})}\cdot
	\fl{A}{\sin(\osnp{e-3})} = \left(\osnp{e-3}\right)^{2} = \osnp{0.1e-5},\\
	\fl{A}{\osnp{1}+\cos(\osnp{e-3})} &=
	\osnp{1} + \fl{A}{\cos(\osnp{e-3})} =
	\osnp{1} + \fl{A}{\osnp{0.999 999 500 000 2}} = 1 + 1 = 2
\end{align*}
(per a $6$ dígits amb arrodoniment), d'on:
\begin{displaymath}
	\fl{A}{f(\osnp{e-5})} = 
	\fl{A}{\frac{\oldstyle{\osnp{0.1e-5}}}{\osnp{0.2e1}}} = \fl{A}{\osnp{0.5e-6}} = 
	\osnp{0.5e-6}.
\end{displaymath}
D'altra banda, per $x$ petit podem fer servir $f(x) = 1 - \cos(x) 
= 2\sin^{2}\left(\dfrac{x}{2}\right)\approx\dfrac{x^{2}}{2}$.
\begin{ex}
	Considereu una fórmula equivalent $f(x) = \osnp{1} -\cos(x) =
	2\sin^{2}\left(\dfrac{x}{2}\right)$ i comproveu que
			$\fl{A}{1-\cos(x)} =
			\fl{A}{2\sin^{2}\left(\dfrac{x}{2}\right)} =
			2\fl{A}%
			{\left(\osnp{0.5e-3}\right)^{2}} = \osnp{0.5e-6}$. 
\end{ex}
\end{solution}

%--------------------------------- Problema 2 --------------------------------
\question\label{prob:2}
Se sap que la sèrie $\sum_{n=1}^{\infty}1/n$ és divergent. No obstant
això, si la intentem ``sumar'' en un ordinador usant precisió simple (per
qüestions de temps) dóna un valor concret. Trobeu aquest valor i expliqueu
aquest fenomen. 
\begin{solution}
	Escrivim el programa \verb^sumhs.cc^ en C/C$++$ amb el codi que es
	mostra al Llistat~\ref{lst:sumhs}. Si el compilem, fent 
\begin{lstlisting}
	g++ sumhs.cc -osumhs
\end{lstlisting}
i l'executem amb la comanda \lstinline^./sumhs^, obtenim: 
\verb^for n =  2097151, sum = 1.54036827e+01^.

\noindent
El que succeeix és que, a mesura que anem sumant, el terme que afegim és més
petit, de manera que per a $N$ suficientment gran,  
\begin{displaymath}
	\fl{}{\sum_{n=1}^{N+1}\frac{1}{n}} =
		\fl{}{\sum_{n=1}^{N}\frac{1}{n}},
\end{displaymath}
($N=\osnp{2097151}$), i el terme $\frac{1}{N+1}$ no afegeix dígits significatius a la
suma. 
\end{solution}
\ifprintanswers
\lstset{style=myc++}
\lstinputlisting[
     caption={\lstinline^sumhs.cc^: ``suma'' de la sèrie harmònica.},
		 label=lst:sumhs,
		 float=t]{sumhs.cc}%{../src/sumhs.cc}
     
\fi

%--------------------------------- Problema 3 --------------------------------
\question\label{prob:3}
Si usem un ordinador que comet errors relatius fitats per $\epsilon$ en la
representació i en les operacions aritmètiques, fiteu l'error comès en el càlcul
de $\sum_{i=1}^{n} x_{i}$.
%
\begin{solution}
(Veure~\cite{AubanellBD91}). Mirem com es propaga l'error relatiu a mesura que
anem afegint termes a la suma:
\begin{enumerate}[label=(\roman*), ref=(\roman*)]
	\item $x_{i}$ es representa com $X_{i} = \fl{}{x_{i}} =
		x_{i}(1+\delta_{i})$, amb $\delta_{i} < \epsilon$,
		$i=1,2,\dots,n$.
	\item $X_{1}+X_{2}$ es representa com
		$S_{2} := \fl{}{X_{1}+X_{2}} = \left(X_{1}+X_{2}\right) 
		\left(1+\delta_{+2}\right)$, amb $|\delta_{+2}|< \epsilon$. Així:
		\begin{align}
			S_{2} =\left[ x_{1}\left(1 + \delta_{1}\right) +
			x_{2}\left(1 + \delta_{2}\right)\right]
			\left(1 + \delta_{+2}\right)
			      &= \left(x_{1} + x_{2} + x_{1}\delta_{1}
				      + x_{2}\delta_{2}\right)\left(1 +
			      \delta_{+2}\right)\nonumber\\
			      &= x_{1} + x_{2} + x_{1}\delta_{1} +
			      x_{2}\delta_{2} + \left(x_{1} +
			      x_{2}\right)\delta_{+2} + \dots,
			      \label{eq:S2}
		\end{align}
		on els punts ($\dots$) denoten termes no lineals en $\delta_{1},
		\delta_{2}, \delta_{+2}$.
	\item $S_{2} + X_{3}$ es representa com $S_{3} = \fl{}{S_{2} + X_{3}}
		=\left(S_{2} + X_{3}\right)\left(1 + \delta_{+3}\right)$, amb
		$|\delta_{+3}| < \epsilon$. Així,
		\begin{align*}
			S_{3} &=\left\{\left[x_{1}\left(1 + \delta_{1}\right) +
				x_{2}\left(1 + \delta_{2}\right)\right]
				\left(1+\delta_{+2}\right) +
				x_{3}\left(1+\delta_{3}\right)\right\}
				\left(1 + \delta_{+3}\right)\\
			      &= x_{1} + x_{2} + x_{3}
			      	+ x_{1}\delta_{1} + x_{2}\delta_{2} 
			      	+ x_{3}\delta_{3}
			      	+ \left(x_{1} + x_{2}\right)\delta_{+2}
			      	+ \left(x_{1} + x_{2} + x_{3}\right)\delta_{+3}
			      	+ \dots
		\end{align*}
%		\begin{align*}
%			S_{3} &=\left[x_{1}\left(1 + \delta_{1}\right) +
%				x_{2}\left(1 + \delta_{2}\right)\right]
%				\left(1+\delta_{+2}\right) +
%			x_{3}\left(1+\delta_{3}\right)\right\}
%			\left(1 + \delta_{+3}\right)\\
%			      &=\left[
%				      \left(x_{1} + x_{2} + x_{1}\delta_{1}
%				      + x_{2}\delta_{2}\right)
%				      \left(1 + \delta_{+2}\right)
%			      + x_{3} + x_{3}\delta_{3}\right]
%			      \left(1 + \delta_{+3}\right)\\
%			      &=
%			      \left(x_{1} + x_{2} + x_{3} + x_{1}\delta_{1}
%				      + x_{2}\delta_{2} + x_{3}\delta_{3}
%				      + \left(x_{1} + x_{2}\right)\delta_{+2}
%			      +\dots\right)\left(1+\delta_{+3}\right)\\
%			      &=
%			      x_{1} + x_{2} + x_{3}
%			      + x_{1}\delta_{1} + x_{2}\delta_{2} 
%			      + x_{3}\delta_{3}
%			      + \left(x_{1} + x_{2}\right)\delta_{+2}
%			      + \left(x_{1} + x_{2} + x_{3}\right)\delta_{+3}
%			      + \dots
%		\end{align*}
		On ara, els punts ($\ldots$) denoten els termes no lineals en
		$\delta_{1}, \delta_{2}, \delta_{3}, \delta_{+2}, \delta_{+3}$.
\end{enumerate}
En general, $S_{i-1}+X_{i}$ es representa com $S_{i} = 
	\fl{}{S_{i-1}+X_{i}} =\left(S_{i-1}+X_{i}\right)\left(1+
	\delta_{+i}\right)$, amb $|\delta_{+i}|<\epsilon$, 
	$i=3,4,\dots,n$ on suposem $n\ge 3$. Per inducció es comprova que, per a
	\begin{align}
		S_{n} &= x_{1} + x_{2} + \dots + x_{n}\nonumber\\
		      &\quad+ x_{1}\delta_{1} + x_{2}\delta_{2} + \dots 
		       + x_{n}\delta_{n}\nonumber\\
		      &\quad+ \left(x_{1} + x_{2}\right)\delta_{+2} 
	               + \left(x_{1} + x_{2} + x_{3}\right)\delta_{+3} 
		       + \dots 
		       +\left(x_{1} + x_{2} + \dots + x_{n}\right)\delta_{+n} 
		       +\dots,\label{eq:Sn}
	\end{align}
	per a $n=2,3,\dots$; i on les e\lgem ipsis ($\ldots$) al final denoten
	els termes no lineals en $\delta_{1}, \delta_{2},\dots,\delta_{n}$;
	$\delta_{+2},\delta_{+3},\dots,\delta_{+n}$. En efecte, per a $n=2$, la
	fórmula~\eqref{eq:Sn} es redueix a la l'equació~\eqref{eq:S2}. Suposem,
	a continuació, que~\eqref{eq:Sn} és cert per a $n = N > 2$ i comprovem
	si amb aquesta hipòtesi (d'inducció), la fórmula~\eqref{eq:Sn} es
	satisfà també per a $n = N+1$:
	\begin{align*}
		S_{N+1} &= \fl{}{S_{N} + X_{N+1}} \\
			&= \left(S_{N} 
		           + X_{N+1}\right)\left(1 + \delta_{+(N+1)}\right)\\
			&= \left[
			        x_{1} + x_{2} + \dots + x_{n} 
			          + x_{1}\delta_{1} + x_{2}\delta_{2} 
				  + \dots + x_{N}\delta_{N}\right.\\
			&\quad+\left(x_{1} + x_{2}\right)\delta_{+2}
				  + \left(x_{1} + x_{2} +
				  x_{3}\right)\delta_{+3}+ \dots\\
			&\quad\left.
				  + \left(x_{1} + x_{2} + \dots 
				  + x_{n}\right)\delta_{+N} 
				  + x_{N+1}\left(1 + \delta_{N+1}\right)
				  +\dots\right]
				  \left(1 + \delta_{+(N+1)}\right)\\
			&= x_{1} + x_{2} + x_{3} + \dots + 
				   x_{N} + x_{N+1}
				 + x_{1}\delta_{1} + x_{2}\delta_{2} 
				+ x_{3}\delta_{3} + \dots 
				+ x_{N}\delta_{N} 
				+ x_{N+1}\delta_{N+1}\\
			&\quad  + \left(x_{1} + x_{2}\right)\delta_{+2}
			        + \left(x_{1} + x_{2} + x_{3}\right)\delta_{+3}
			        + \left(x_{1} + x_{2} + x_{3} 
				        + x_{4}\right)\delta_{+4}
				+ \dots\\
			&\quad  + \left(x_{1} + x_{2} + x_{3} +\dots 
					+ x_{N}\right)\delta_{+N}
					+ \left(x_{1} + x_{2} + x_{3}+\dots 
						+ x_{N} + x_{N+1}\right)
						\delta_{+(N+1)} + \dots
	\end{align*}
	Això prova~\eqref{eq:Sn} per a tot $n\in\N$ amb $n\ge 2$. Si
	a~\eqref{eq:Sn} restem $\sum_{i=1}^{n}x_{i}$, agafem valor absolut i acotem
	$|\delta_{+i}| < \epsilon$, per a $i=2,3,\dots,n$ tenim:
	\begin{displaymath}
		\left|S_{n} - \sum\limits_{i=1}^{n}x_{i}\right| \le
		\sum_{i=1}^{n} |x_{i}||\delta_{i}| + 
		\left[(n-1)|x_{1}| + (n-1)|x_{2}| +
		(n-2)|x_{3} + \dots + 2|x_{n-1}| + |x_{n}|\right]
		\epsilon + \dots,
	\end{displaymath}
	on els punts denoten ara els termes no lineals en $\delta_{1}$,
	$\delta_{2},\dots,\delta_{n}$; $\epsilon$. Si a continuació, com abans,
	acotem $|\delta_{i}|\le\epsilon$, per a $i=1,2,\dots,n$; tenim:
	\begin{displaymath}
		\left|S_{n} - \sum\limits_{i=1}^{n}x_{i}\right| \le
		\left(n|x_{1}| + n|x_{2}| + (n-1)|x_{3}| 
		+ \dots + 3|x_{n-1}| + 2|x_{n}|\right)\epsilon + \dots,
	\end{displaymath}
	on els punts indiquen termes d'ordre superior en $\epsilon$
	($\epsilon^{2}$, $\epsilon^{3},\dots$). Si suposem que podem prescindir
	d'aquests termes que ---procedeixen dels termes no lineals en els errors
	$\delta_{1}$, $\delta_{2},\delta,\delta_{n}$; $\delta_{+2}$,
	$\delta_{+3},\dots,\delta_{+n}$---s'obté la següent fita per a l'error  
	absolut de la suma,
	\begin{equation}\label{eq:BoundErrorS}
		\left|S_{n} - \sum\limits_{i=1}^{n}x_{i}\right| \le
		\left(n|x_{1}| + n|x_{2}| + (n-1)|x_{3}| 
		+ \dots + 3|x_{n-1}| + 2|x_{n}|\right)\epsilon. 
	\end{equation}
	En conseqüència, com s'assenyala a~\cite{AubanellBD91}, pàg.~35,
	aquesta cota es pot minimitzar ``si es sumen els termes en ordre
	creixent de valor absolut, començant amb els de valor absolut més petit
	i acabant amb els de valor absolut més gran''.
\end{solution}

%--------------------------------- Problema 4 --------------------------------
\question\label{prob:4}
Useu aritmètica de $3$ dígits amb tall per a calcular la suma 
$\sum_{i=1}^{15} 1/i^{2}$ primer en l'ordre natural, 
$\osnp{1} + \osnp{1}/\osnp{4} + \dots + \osnp{1}/\osnp{225}$,
i després a l'inrevés, $\osnp{1}/\osnp{225} + \osnp{1}/\osnp{196} +\dots +
\osnp{1}/\osnp{4} + \osnp{1}$. Decidiu quin és el mètode més exacte de tots dos. 
\begin{solution}
Com que treballem amb $3$ dígits amb tall, $\epsilon <
\osnp{10}^{\osnp{1}-\osnp{3}}=\osnp{e-2}$. Llavors si $S_{15}$ és la suma
que s'obté en l'ordre natural (decreixent) i $\bar{S}_{15}$ la que obtenim
sumant en ordre creixent de termes, i.e., $\osnp{1}/\osnp{255} + 
\osnp{1}/\osnp{196} + \dots + \osnp{1}/\osnp{4} + \osnp{1}$, aplicant la
fórmula~\eqref{eq:BoundErrorS} del problema~\ref{prob:3} resulta,
\begin{align*}
	S_{15} = \osnp{0.153e1},&\
	\left|S_{15} - \left(\osnp{1} +
	\frac{\osnp{1}}{\osnp{4}} + \frac{\osnp{1}}{\osnp{9}}+
	\dots + \frac{\osnp{1}}{\osnp{196}} +
	 \frac{\osnp{1}}{\osnp{255}}\right)\right|\\
        &\quad\le\left(\osnp{15}\cdot\osnp{1} +
	\osnp{15}\cdot\frac{\osnp{1}}{\osnp{2}^{\osnp{2}}} + 
	\osnp{14}\cdot\frac{\osnp{1}}{\osnp{3}^{\osnp{2}}} +\dots
	+ \osnp{3}\cdot\frac{\osnp{1}}{\osnp{14}^{\osnp{2}}}
	+ \osnp{2}\frac{1}{\cdot{15}^{\osnp{2}}}\right)
	\cdot\osnp{e-2} = \osnp{0.22549},\\
		\bar{S}_{15} = \osnp{0.157e1},&\ 
		\left|\bar{S}_{15} - \left(\osnp{1} +
		\frac{\osnp{1}}{\osnp{4}} + \frac{\osnp{1}}{\osnp{9}}+
		\dots +	\frac{\osnp{1}}{\osnp{196}} +
		 \frac{\osnp{1}}{\osnp{255}}\right)\right|\\
	       &\quad\le 
	\left(\osnp{15}\cdot\frac{\osnp{1}}{\osnp{15}^{\osnp{2}}} +
	      \osnp{15}\cdot\frac{\osnp{1}}{\osnp{14}^{\osnp{2}}} +
	      \osnp{14}\cdot\frac{\osnp{1}}{\osnp{13}^{\osnp{2}}} +\dots +
	      \osnp{3}\cdot\frac{\osnp{1}}{\osnp{2}^{\osnp{2}}} +
      \osnp{2}\cdot\osnp{1}\right)\osnp{e-2} = \osnp{0.04894}.
	\end{align*}
Fent el mateix càlcul amb $\osnp{16}$ dígits significatius ($t =
\osnp{16}$) es troba, per a la suma en ordre creixent, $\bar{S}_{15} =
\osnp{1.580440283}\approx\osnp{0.158e1}$.
\end{solution}

%--------------------------------- Problema 5 --------------------------------
\question\label{prob:5}
Demostreu que en l'operació $\sqrt{x}$ l'error relatiu és aproximadament la
meitat de l'error relatiu en les dades. Direm que l'operació de fer $\sqrt{x}$
és una operació \emph{segura} respecte de l'error relatiu. Feu patent la
``inseguretat'' de l'operació $f(x) = \dfrac{1}{1-x^{2}}$ per a $x\approx\osnp{1}$.
(Suposeu que tenim errors només en la representació de les dades). 
\begin{solution}
%  
%	Solució del problema~\ref{prob:5}.  
%
Definim
\begin{displaymath}
 e_{a}\left(f\left(\tilde{x}\right)\right) := f\left(\tilde{x}\right) -
 f\left(x\right),\qquad
  e_{r}\left(f\left(\tilde{x}\right)\right) := \frac{f\left(\tilde{x}\right) -
  f\left(x\right)}{f\left(x\right)},
\end{displaymath}
on suposem que $x \ne 0, f(x)\ne 0$, i que $f$ té derivada contínua en algun interval
obert, $J_{x}$, $x\in J_{x}, 0\notin J_{x}$, on $f$ no s'anul·la. 
%Considerem
%$\tilde{x}$ un punt de l'interval $J_{x}$ proper a $x$.
Llavors, si $\epsilon_{a}\left(\tilde{x}\right)$ i
$\epsilon_{r}\left(\tilde{x}\right)$ són fites dels valors absoluts dels
errors absolut i relatiu de $x$ respectivament, tals que 
\begin{equation}\label{eq:sigma-eaar}
  \left|e_{a}\left(\tilde{x}\right)\right| := \left|\tilde{x} - x\right|
  \le \sigma\epsilon_{a}\left(\tilde{x}\right),\qquad
  \left|e_{r}\left(\tilde{x}\right)\right| := \left|\frac{\tilde{x} - x}{x}\right|
  \le \sigma\epsilon_{r}\left(\tilde{x}\right)
\end{equation}
par algun $0 < \sigma < 1$, aleshores 
%Llavors, si
%$\epsilon_{a}\left(f\left(\tilde{x}\right)\right)$ i
%$\epsilon_{r}\left(f\left(\tilde{x}\right)\right)$ denoten fites dels
%valors absoluts dels errors $e_{a}\left(f\left(\tilde{x}\right)\right)$ i
%$e_{r}\left(f\left(\tilde{x}\right)\right)$ respectivament, i.e,
%Aleshores,
\begin{enumerate}[label=(\roman*), ref=(\roman*)]
\item \label{item:p-err-1}  $\epsilon_{a}\left(f\left(\tilde{x}\right)\right) 
  :=\left|f'\left(\tilde{x}\right)\right|\epsilon_{a}\left(\tilde{x}
  \right)$,
%\end{enumerate}
%i, si $f(x)\ne 0$,
%\begin{enumerate}[resume*]
\item \label{item:p-err-2} 
  $\epsilon_{r}\left(f\left(\tilde{x}\right)\right) := \left|
  \dfrac{f'\left(\tilde{x}\right)}{f\left(\tilde{x}\right)}\right|
  \epsilon_{r}\left(\tilde{x}\right)$,
\end{enumerate}
són fites dels valors absoluts de
$e_{a}\left(f\left(\tilde{x}\right)\right)$ i
$e_{r}\left(f\left(\tilde{x}\right)\right)$, i.e., 
\begin{equation}\label{eq:fites-ea-er}
  \left|e_{a}\left(f\left(\tilde{x}\right)\right)\right| \le
  \epsilon_{a}\left(f\left(\tilde{x}\right)\right),\qquad
  \left|e_{r}\left(f\left(\tilde{x}\right)\right)\right| \le
  \epsilon_{r}\left(f\left(\tilde{x}\right)\right).
\end{equation}
quan $\tilde{x}$ està suficientment a prop de $x$. En efecte,
%
\begin{enumerate}[label=(\roman*), ref=(\roman*)]
  \item Aplicant el desenvolupament de Taylor, 
\begin{displaymath}
  f\left(\tilde{x}\right):= f(x) + f'\left(\tilde{x}\right)
  \left(\tilde{x}-x\right) 
  + \gamma\left(\tilde{x}\right)\left(\tilde{x}-x\right)
\end{displaymath}
%\begin{displaymath}  
%  \left|e_{a}\left(f\left(\tilde{x}\right)\right)\right| :=
%    \left|f\left(\tilde{x}\right) - f\left(x\right)\right| =
%    \left|f'\left(\xi_{x}\right)\right|\left|\tilde{x} - x\right|,
%\end{displaymath}
amb $\gamma\left(\tilde{x}\right) \to 0$ quan $\tilde{x}\to x$. 
Per tant, 
\begin{equation}\label{eq:eafx}
  e_{a}\left(f\left(\tilde{x}\right)\right) := f\left(\tilde{x}\right) -
  f(x) = f'\left(\tilde{x}\right)\left(1 +
    \gamma\left(\tilde{x}\right)\right)e_{a}\left(\tilde{x}\right) 
\end{equation}
%amb $\xi_{x}\in\langle x, \tilde{x}\rangle$, on $\langle x,
%\tilde{x}\rangle = \left[x,\tilde{x}\right]$, si $\tilde{x} > x$, 
%amb $\xi_{x}\in\langle x, \tilde{x}\rangle$, on $\langle x,
%\tilde{x}\rangle = \left[x,\tilde{x}\right]$, si $\tilde{x} < x$.
%Aleshores,
%\begin{displaymath}
%  \left|e_{a}\left(f\left(\tilde{x}\right)\right)\right| =
%  \left|f'\left(\xi_{x}\right)\right|\left|\tilde{x} - x\right| =
%  \left|f'\left(\xi_{x}\right)\right|\left|e_{a}\left(\tilde{x}\right)\right| \le
%    \left|f'\left(\xi_{x}\right)\right|\epsilon_{a}\left(\tilde{x}\right)
%    =: \epsilon_{a}\left(f\left(\tilde{x}\right)\right).
%\end{displaymath}
d'on tenim
\begin{displaymath}
  \left|e_{a}\left(f\left(\tilde{x}\right)\right)\right| = 
  \left|f'\left(\tilde{x}\right)\right|
  \left|e_{a}\left(\tilde{x}\right)
  \right|\left| 1 + \gamma\left(\tilde{x}\right)\right|\le
  \left|f'\left(\tilde{x}\right)\right|
  \sigma\epsilon_{a}\left(\tilde{x}\right)
  \left| 1 + \gamma\left(\tilde{x}\right)\right| 
  = \sigma\left(1 + \left|\gamma\left(\tilde{x}\right)\right|\right)
  \epsilon_{a}\left(f\left(\tilde{x}\right)\right)
\end{displaymath}
i, donat que $\gamma\left(\tilde{x}\right)\to 0$ quan $\tilde{x}\to x$, 
es conclou que si $\tilde{x}$ és suficientment proper a $x$,
$\epsilon_{a}\left(f\left(\tilde{x}\right)\right)$ és una fita 
superior del valor absolut de $e_{a}\left(f\left(\tilde{x}\right)\right)$,
la qual cosa prova la primera de les desigualtats a~\eqref{eq:fites-ea-er}.
%
%Si $|e_{a}\left(\tilde{x}\right)| := \left|\tilde{x}-
%x\right|$ és petit (escrivim $\left|e_{a}\left(\tilde{x}\right)\right|\ll
%1$), llavors també $\left|\xi_{x} - \tilde{x}\right| \le
%\left|e_{a}\left(\tilde{x}\right)\right|\ll 1$ 
%i, com que
%$f'$ és contínua, tenim que $f'\left(\xi_{x}\right)\approx
%f'\left(\tilde{x}\right)$. Per tant:
%\begin{displaymath}
%  \epsilon_{a}\left(f\left(\tilde{x}\right)\right) :=
%  \left|f'\left(\xi_{x}\right)\right|\epsilon_{a}\left(\tilde{x}\right)\approx
%  \left|f'\left(\tilde{x}\right)\right|\epsilon_{a}\left(\tilde{x}\right).
%\end{displaymath}
\item Fent servir~\eqref{eq:eafx} i les propietats de $f$ a l'entorn 
$J_{x}$ del punt $x$ descrit adalt es comprova (exercici!),
\begin{equation}\label{eq:erfx}
  e_{r}\left(f\left(x\right)\right) = \left(1 +
    \zeta\left(\tilde{x}\right)\right)
    \frac{f'\left(\tilde{x}\right)}{f\left(\tilde{x}\right)}e_{a}
    \left(\tilde{x}\right)
\end{equation}
amb $\zeta\left(\tilde{x}\right)\to 0$ quan $\tilde{x}\to x$.
D'altra banda, per definició, d'error relatiu,
\begin{displaymath}
    e_{a}\left(\tilde{x}\right) := xe_{r}\left(\tilde{x}\right) =
    \tilde{x}\left(1 -
    \frac{\tilde{x}-x}{\tilde{x}}\right)e_{r}\left(\tilde{x}\right).
\end{displaymath}
Si es substitueix aquesta última espressió per a
$e_{a}\left(\left(\tilde{x}\right)\right)$ a~\eqref{eq:erfx} s'obté
\begin{equation}\label{eq:erfx-1}
  e_{r}\left(f\left(\tilde{x}\right)\right) = \left(1 +
  \eta\left(\tilde{x}\right)\right)\frac{f'\left(\tilde{x}\right)\tilde{x}}
  {f\left(\tilde{x}\right)}e_{r}\left(\tilde{x}\right)
\end{equation}
amb $\eta\left(\tilde{x}\right)\to 0$ quan $\tilde{x}\to x$. Per últim,
prenent valors absoluts a l'equació~\eqref{eq:erfx-1} i acotant fent
servir~\eqref{eq:sigma-eaar} s'arriba a,
\begin{align*}
  \left|e_{r}\left(f\left(\tilde{x}\right)\right)\right| &=
  \left| 1 + \eta\left(\tilde{x}\right) \right|
  \left|\frac{f'\left(\tilde{x}\right)\tilde{x}}
  {f\left(\tilde{x}\right)}\right|\left|e_{r}\left(\tilde{x}\right)\right|
  \le
  \sigma\left( 1 + \left|\eta\left(\tilde{x}\right)\right| \right)
  \left|\frac{f'\left(\tilde{x}\right)\tilde{x}}
  {f\left(\tilde{x}\right)}\right|\epsilon_{r}\left(\tilde{x}\right)\\
  &= \sigma\left(1 + \left|\eta\left(\tilde{x}\right)\right| \right)
  \epsilon_{r}\left(f\left(\tilde{x}\right)\right)
\end{align*}
i, atès que $\eta\left(\tilde{x}\right)\to 0$ quan $\tilde{x}\to 0$, es
conclou, amb els mateixos arguments que a~\ref{item:p-err-1}, que  
si $\tilde{x}$ és suficientment proper a $x$,
$\epsilon_{r}\left(f\left(\tilde{x}\right)\right)$ és una fita 
superior del valor absolut de $e_{r}\left(f\left(\tilde{x}\right)\right)$,
la qual cosa prova la segona de les desigualtats a~\eqref{eq:fites-ea-er}.
\end{enumerate}
Considerem $f(x) = \sqrt{x}$, en un interval $I$ que no contingui el $0$.
Sigui $x\in I$ el valor exacte i $\tilde{x}\in I$ una aproximació a $x$.
D'acord amb les definicions~\ref{item:p-err-1}, \ref{item:p-err-2},
\end{solution}

%--------------------------------- Problema 6 --------------------------------
\question\label{prob:6}
Donat el sistema d'equacions lineals
\begin{align*}
	3\,x + a\,y &= 10\\
	5\,x + b\,y &= 20
\end{align*}
on $a = \osnp{2.100}\osnp{+-5e-4}$ i $b = \osnp{3.300}\osnp{+-5e-4}$; amb 
quina exactitud pot ser determinat $x+y$? (Suposant operacions exactes).
\begin{solution}
	Solució del problema~\ref{prob:6}.
\end{solution}

%--------------------------------- Problema 7 --------------------------------
\question\label{prob:7}
Treballant amb $5$ xifres decimals, calculeu
\begin{displaymath}
	\sqrt[k]{\osnp{2.15283}} - \sqrt[k]{\osnp{2.15263}},\qquad\qquad
	k=2,3,4.
\end{displaymath}
\begin{parts}
	\part Directament.
	\part Usant fórmules millors des del punt de vista numèric. (Indicació:
		Feu la divisió de polinomis $(a^{k} - b^{k})/(a-b)$).
	\part Compareu els resultats i comenteu-los.
\end{parts}
\begin{solution}
	Solució del problema~\ref{prob:7}.
\end{solution}

%\question\label{prob:8}
%Trobeu una expressió de la fita approximada de l'error absolut en avaluar 
%la funció  $f(x) = \rme^{x^{2}}$, on es suposa que els erros relatius en la
%representació dels nombres, en el producte i en l'exponecial són, en valor
%absolut, menors respectivament que $\epsilon_{r}$, $\epsilon_{*}$ i 
%$\epsilon_{\exp}$. 
%\begin{solution}
%	Solució del problema~\ref{prob:8}.
%\end{solution}

%--------------------------------- Problema 8 --------------------------------
\question\label{prob:8}
Es vol calcular $f_{n}(x)$, on $f_{n}(x) = n!\left[\rme^{x} -
	\left(1 + x + \frac{x^{2}}{2!} + \dots + \frac{x^{n}}{n!}\right)
\right]$, per a $x = \osnp{1}$, $n=\osnp{0},\osnp{1},\osnp{2},\dots$
\begin{parts}
	\part Demostreu que se satisfà la següent llei de recurrència
	\begin{displaymath}
		f_{n+1}(x) = (n+1) f_{n}(x) - x^{n+1},
	\end{displaymath}
	\part Calculeu $f_{n}(1)$, $n=1\div 10$ amb aritmètica de coma
	flotant amb $5$ dígits significatius. És fiable el resultat?
	Què es pot fer per a calcular $f_{5}(1)$?
\end{parts}
\begin{solution}
	Solució del prooblema~\ref{prob:8}.
\end{solution}

%--------------------------------- Problema 9 --------------------------------
\question\label{prob:9}
Usant un mètode recurrent, calculeu el valor de les integrals
\begin{displaymath}
	I_{j} = \int_{0}^{1} x^{j}\sin(\pi x)\,\rmd x,
\end{displaymath}
($j=2,4,\dots,20$). Estudieu l'estabilitat del mètode trobat.
\textbf{Nota:} en general, direm que un mètode és \emph{estable} si esmorteeix
els errors fets en les aproximacions. Quan, pel contrari, els amplifica, direm
que el mètode és \emph{inestable}.
\begin{solution}
	Solució del problema~\ref{prob:9}
\end{solution}

%--------------------------------- Problema 10 -------------------------------
\question\label{prob:10}
Volem calcular
$a = \left(\osnp{7} - \osnp{4}\sqrt{\osnp{3}}\right)^{\osnp{4}}$
emprant el valor aproximat $\osnp{1.73205}$ per a $\sqrt{\osnp{3}}$.
Prenem dues fórmules per a calcular $a$:

\begin{oneparchoices}
	\choice $\dfrac{\osnp{1}}{\left(\osnp{7} + 
	\osnp{4}\sqrt{\osnp{3}}\right)^{4}}$, 
	\choice $\dfrac{\osnp{1}}{\osnp{18817} +\osnp{10864}
	\sqrt{\osnp{3}}}$.
\end{oneparchoices}

Quina és la més adequada des del punt de vista numèric (suposant que
les operacions es fan sense errors)?
\begin{solution}
	Solució del problema~\ref{prob:10}.
\end{solution}

%--------------------------------- Problema 11 -------------------------------
\question\label{prob:11}
Doneu una expressió equivalent per a cadascuna de les fórmules següents
que sigui millor des del punt de vista numèric, quan $\varepsilon$ és
molt més petit que $x$.
\begin{parts}
	\part $\dfrac{1}{\sqrt{x-\varepsilon}} - \dfrac{1}{\sqrt{x}}$.
	\part $\sin(x+\varepsilon) - \sin(x)$.
	\part $\cos(x+\varepsilon) - \cos(x)$.
	\part $\displaystyle\int_{x}^{x+\varepsilon}\dfrac{\rmd t}{t}$.
\end{parts}
\begin{solution}
	Solució del problema~\ref{prob:11}. 
\end{solution}

%--------------------------------- Problema 12 -------------------------------
\question\label{prob:12}
Amb quina exactitud s'ha de mesurar el radi d'una esfera i amb quants 
decimals cal donar el nombre $\pi$ perquè el seu volum es conegui amb 
un error relatiu menor que el $\osnp{0.01}$\%? Considereu ambdós 
efectes per separat.
\begin{solution}
	Solució del problema~\ref{prob:12}.
\end{solution}
\end{questions}

%-------------------------------- Referències --------------------------------
\ifprintanswers 
\bibliographystyle{plain}
%\nocite{*}
\bibliography{alnRefs}
\fi

\end{document}
