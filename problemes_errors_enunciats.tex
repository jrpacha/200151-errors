%%%%%%%%%%%%%%%%%%%%%%%%%%%%%%%%%%%%%%%%%%%%%%%%%%%%%%%%%%%%%%%%%%%%%%%%%%%%%%
% Tema 1. Problemes. Aritmètica Finita i Control d'Errors
%%%%%%%%%%%%%%%%%%%%%%%%%%%%%%%%%%%%%%%%%%%%%%%%%%%%%%%%%%%%%%%%%%%%%%%%%%%%%%
\documentclass[a4paper,twoside,12pt]{exam}
\usepackage{problemes}

\def\assignatura{\`{A}lgebra Lineal Num\`{e}rica}
\def\codi{200151}
\def\GrauMaster{MAT}
\def\EscolaFacultat{FME}

\hypersetup{
  pdfauthor   = {Equip Docent ALN},
  pdftitle    = {Problemes Tema 1: Arim\`{e}tica Finita i Control d'Errors},
  pdfsubject  = {Problemes}
}

\renewcommand{\solutiontitle}%
{\noindent\textbf{Solució:}\enspace}

\pagestyle{head}
\firstpageheader{\EscolaFacultat}%
		{\assignatura~(\codi)}%
		{\thepage}%
\firstpageheadrule

\runningheader{}{\thesection.~\currentname}{\thepage}
\runningheadrule

\bracketedpoints
\unframedsolutions

\newcommand{\osnp}[1]{\oldstyle{\np{#1}}}
\npdecimalsign{.}
\npthousandsep{\,}
\npproductsign{\times}%

\renewcommand{\choicelabel}{(\roman{choice})}

\begin{document}
\section{Aritmètica finita i control d'errors}
%\setcounter{problema}{0}

\begin{questions}
%--------------------------------- Problema 1 --------------------------------
\question\label{prob:1}
Calculeu $f(x) = 1 - \cos(x)$ amb aritmètica de coma flotant de $\osnp{6}$
dígits, per a $|x| < \osnp{e-3}$. És fiable el resultat? Feu el mateix amb la
representació de $f(x)$ 
\begin{displaymath}
	f(x) = \frac{\sin^{2}(x)}{1 + \cos(x)}.
\end{displaymath}
Trobeu una altra representació alternativa per a $f(x)$ que sigui vàlida per a
$|x|$ petit.
\begin{solution}
	Solució del problema~\ref{prob:1}.
\end{solution}

%--------------------------------- Problema 2 --------------------------------
\question\label{prob:2}
Se sap que la sèrie $\sum_{n=1}^{\infty}1/n$ és divergent. No obstant
això, si la intentem ``sumar'' en un ordinador usant precisió simple (per
qüestions de temps) dóna un valor concret. Trobeu aquest valor i expliqueu
aquest fenomen. 
\begin{solution}
	Solució del problema~\ref{prob:2}.
\end{solution}

%--------------------------------- Problema 3 --------------------------------
\question\label{prob:3}
Si usem un ordinador que comet errors relatius fitats per $\epsilon$ en la
representació i en les operacions aritmètiques, fiteu l'error comès en el càlcul
de $\sum_{i=1}^{n} x_{i}$.
\begin{solution}
	Solució del problema~\ref{prob:3}.
\end{solution}

%--------------------------------- Problema 4 --------------------------------
\question\label{prob:4}
Useu aritmètica de $3$ dígits amb tall per a calcular la suma 
$\sum_{i=1}^{15} 1/i^{2}$ primer en l'ordre natural, 
$\osnp{1} + \osnp{1}/\osnp{4} + \dots + \osnp{1}/\osnp{225}$,
i després a l'inrevés, $\osnp{1}/\osnp{225} + \osnp{1}/\osnp{196} +\dots +
\osnp{1}/\osnp{4} + \osnp{1}$. Decidiu quin és el mètode més exacte de tots 
dos. 
\begin{solution}
	Solució del problema~\ref{prob:4}.
\end{solution}

%--------------------------------- Problema 5 --------------------------------
\question\label{prob:5}
Demostreu que en l'operació $\sqrt{x}$ l'error relatiu és aproximadament la
meitat de l'error relatiu en les dades. Direm que l'operació de fer $\sqrt{x}$
és una operació \emph{segura} respecte de l'error relatiu. Feu patent la
``inseguretat'' de l'operació $f(x) = \frac{1}{1-x^{2}}$ per a $x\simeq\osnp{1}$.
(Suposeu que tenim errors només en la representació de les dades). 
\begin{solution}
	Solució del problema~\ref{prob:5}.
\end{solution}

%--------------------------------- Problema 6 --------------------------------
\question\label{prob:6}
Donat el sistema d'equacions lineals
\begin{align*}
	3\,x + a\,y &= 10\\
	5\,x + b\,y &= 20
\end{align*}
on $a = \osnp{2.100}\osnp{+-5e-4}$ i $b = \osnp{3.300}\osnp{+-5e-4}$; amb quina
exactitud pot ser determinat $x+y$? (Suposant operacions exactes).
\begin{solution}
	Solució del problema~\ref{prob:6}.
\end{solution}

%--------------------------------- Problema 7 --------------------------------
\question\label{prob:7}
Treballant amb $5$ xifres decimals, calculeu
\begin{displaymath}
	\sqrt[k]{\osnp{2.15283}} - \sqrt[k]{\osnp{2.15263}},\qquad\qquad
	k=2,3,4.
\end{displaymath}
\begin{parts}
	\part Directament.
	\part Usant fórmules millors des del punt de vista numèric. (Indicació:
		Feu la divisió de polinomis $(a^{k} - b^{k})/(a-b)$).
	\part Compareu els resultats i comenteu-los.
\end{parts}
\begin{solution}
	Solució del problema~\ref{prob:7}.
\end{solution}

%--------------------------------- Problema 8 --------------------------------
\question\label{prob:8}
Es vol calcular $f_{n}(x)$, on $f_{n}(x) = n!\left[\rme^{x} -
	\left(1 + x + \frac{x^{2}}{2!} + \dots + \frac{x^{n}}{n!}\right)
\right]$, per a $x = \osnp{1}$, $n=\osnp{0},\osnp{1},\osnp{2},\dots$
\begin{parts}
	\part Demostreu que se satisfà la següent llei de recurrència
	\begin{displaymath}
		f_{n+1}(x) = (n+1) f_{n}(x) - x^{n+1},
	\end{displaymath}
	\part Calculeu $f_{n}(1)$, $n=1\div 10$ amb aritmètica de coma
	flotant amb $5$ dígits significatius. És fiable el resultat?
	Què es pot fer per a calcular $f_{5}(1)$?
\end{parts}
\begin{solution}
	Solució del prooblema~\ref{prob:8}.
\end{solution}

%--------------------------------- Problema 9 --------------------------------
\question\label{prob:9}
Usant un mètode recurrent, calculeu el valor de les integrals
\begin{displaymath}
	I_{j} = \int_{0}^{1} x^{j}\sin(\pi x)\,\rmd x,
\end{displaymath}
($j=2,4,\dots,20$). Estudieu l'estabilitat del mètode trobat.
\textbf{Nota:} en general, direm que un mètode és \emph{estable} si esmorteeix
els errors fets en les aproximacions. Quan, pel contrari, els amplifica, direm
que el mètode és \emph{inestable}.
\begin{solution}
	Solució del problema~\ref{prob:10}
\end{solution}

%--------------------------------- Problema 10 -------------------------------
\question\label{prob:10}
Volem calcular
$a = \left(\osnp{7} - \osnp{4}\sqrt{\osnp{3}}\right)^{\osnp{4}}$
emprant el valor aproximat $\osnp{1.73205}$ per a $\sqrt{\osnp{3}}$.
Prenem dues fórmules per a calcular $a$:

\begin{oneparchoices}
	\choice $\dfrac{\osnp{1}}{\left(\osnp{7} + 
	\osnp{4}\sqrt{\osnp{3}}\right)^{4}}$, 
	\choice $\dfrac{\osnp{1}}{\osnp{18817} +\osnp{10864}
	\sqrt{\osnp{3}}}$.
\end{oneparchoices}

Quina és la més adequada des del punt de vista numèric (suposant que
les operacions es fan sense errors)?
\begin{solution}
	Solució del problema~\ref{prob:10}.
\end{solution}

%--------------------------------- Problema 11 -------------------------------
\question\label{prob:11}
Doneu una expressió equivalent per a cadascuna de les fórmules següents
que sigui millor des del punt de vista numèric, quan $\varepsilon$ és
molt més petit que $x$.
\begin{parts}
	\part $\dfrac{1}{\sqrt{x-\varepsilon}} - \dfrac{1}{\sqrt{x}}$.
	\part $\sin(x+\varepsilon) - \sin(x)$.
	\part $\cos(x+\varepsilon) - \cos(x)$.
	\part $\displaystyle\int_{x}^{x+\varepsilon}\dfrac{\rmd t}{t}$.
\end{parts}
\begin{solution}
	Solució del problema~\ref{prob:11}. 
\end{solution}

%--------------------------------- Problema 12 -------------------------------
\question\label{prob:12}
Amb quina exactitud s'ha de mesurar el radi d'una esfera i amb quants 
decimals cal donar el nombre $\pi$ perquè el seu volum es conegui amb 
un error relatiu menor que el $\osnp{0.01}$\%? Considereu ambdós 
efectes per separat.
\begin{solution}
	Solució del problema~\ref{prob:12}.
\end{solution}
\end{questions}

\end{document}
